\documentclass{article}

\usepackage[latin1]{inputenc}
\usepackage[brazil]{babel}
\usepackage{tikz}
\usetikzlibrary{shapes,arrows}

%%%<
\usepackage{verbatim}
\usepackage[active,tightpage]{preview}
\PreviewEnvironment{tikzpicture}
\setlength\PreviewBorder{5pt}%
%%%>

\begin{comment}
:Title: Simple flow chart
:Tags: Diagrams

With PGF/TikZ you can draw flow charts with relative ease. This flow chart from [1]_
outlines an algorithm for identifying the parameters of an autonomous underwater vehicle model. 

Note that relative node
placement has been used to avoid placing nodes explicitly. This feature was
introduced in PGF/TikZ >= 1.09.

.. [1] Bossley, K.; Brown, M. & Harris, C. Neurofuzzy identification of an autonomous underwater vehicle `International Journal of Systems Science`, 1999, 30, 901-913 


\end{comment}


\begin{document}
\pagestyle{empty}


% Define block styles
\tikzstyle{decision} = [diamond, draw, fill=blue!20, 
    text width=10em, text badly centered, node distance=3.5cm, inner sep=0pt]
\tikzstyle{block} = [rectangle, draw, fill=green!20, 
    text width=30em, rounded corners, node distance=3.5cm, minimum height=4em]
\tikzstyle{line} = [draw, -latex']
\tikzstyle{cloud} = [draw, ellipse,fill=red!20, 
    text width=5em, text centered, node distance=5cm,
    minimum height=2em]
    
\begin{tikzpicture}[node distance = 2cm, auto]
    % Place nodes
    \node [decision] (dados) {{\ttfamily Dados}};
    \node [cloud, left of=dados] (epi) {Quest\~{o}es epidemiol\'{o}gicas};
    \node [cloud, right of=dados] (estat) {An\'{a}lise estat\'{i}stica};
    \node [decision, right of=estat] (implementa) {{\ttfamily Implementa��o}};
    \node [cloud, right of=implementa] (r) {\includegraphics[width=\columnwidth]{Rlogo}};
		\path [line] (epi) -- (dados);
		\path [line] (dados) -- (estat);
		\path [line] (estat) -- (implementa);
    \path [line] (implementa) -- (r);
    %\path [line,dashed] (system) |- (evaluate);
\end{tikzpicture}

\end{document}